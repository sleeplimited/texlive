\documentclass{article}
%    version = 4.03 of businesscard.tex 2012 Apr 29
% 2012 Apr 29, 4.03: rename from homecard.tex back to businesscard.tex
% 2012 Apr 29, 4.02: rename from businesscard.tex to homecard.tex
% 2009 Nov 27, 4.01: multiput the cards now
% 2009 Nov 26, 4.00: revamp completely using pstricks, geometry packages
% 2009 Nov 26, 3.16: ?? - can't solve positioning exactly
% 2009 Nov 26, 3.15: adjust to fit Staples(R) brand business cards
% 2009 Nov 26, 3.15: adjust to fit standard Tex installation on Mac
% 2006 Aug 01, 3.14: adjust topmargin to fit A4 paper
% 2006 May 14, 3.13: clean up
% 2006 May 14, 3.12: test limits of page - 5 per page works!
% 2006 May 14, 3.11: proper order of pstricks call
% 2006 May 13, 3.10: drop pstcol package, it's old!
% 2006 May 13, 3.09: switch from the color to the pstcol package
% 2006 May 13, 3.08: new command verticalcards determines number of cards
% 2006 May 12, 3.07: more documentation
% 2006 May 12, 3.06: cleanup
% 2006 May 12, 3.05: rename from workcard.tex to businesscard.tex
% 2006 May 12, 3.04: brush up
% 2005 Dec 31, 3.03: functional

% This is a LaTeX file for creating a business card.

% You just supply a 'guts.tex' file and it makes a set of 10 copies
% per page that you can cut out. I print it on plain paper and
% although the cards are not stiff, people accept them anyway!  The
% advantage is that one can always print more, one can change them
% quickly and they don't cost much.

% This revised version uses standard spacing and so can be printed on
% pre-perferated card stock paper.

% Two variables, xadjust and yadjust can be set to position the cards
% exactly onto card stock.  Print on regular paper, With a bright
% light compare to the stock and measure the change in cm required to
% determine the adjustments.  Be sure that the papers are aligned at
% the feeding edge since they can be slightly different sizes.

% This may not be true anymore:
% With the initial values, businesscard.tex works for both letter and
% a4 paper, giving 5 cards vertically and 2 horizontally.

% This file reads a file called 'guts.tex'.
% Use \\ between lines of the guts.
% You can use pstricks graphics.
%
% Example name with date:
% Tom Schneider \ldots \ldots \ldots \ldots 2005 Sep 7 \\
% (Note: \dotfill might work nicely, I haven't tried it yet.)
%
% Example to input a file:
% \input workurl.tex
%
% Example vertical space, move down:
% \vspace{3pt} \\
%
% Example vertical space, move up:
% \vspace{-3pt} \\
%
% Example graphic:
% \rotatebox{0}{\scalebox{0.60}{\includegraphics*{something.eps}}}
%
% source:
%
% http://alum.mit.edu/www/toms/latex.html#businesscard
%
% Dr. Thomas D. Schneider
% National Institutes of Health
% schneidt@mail.nih.gov
% toms@alum.mit.edu (permanent)
% http://alum.mit.edu/www/toms (permanent)

% NOTES

% Note: I have inserted '\usepackage{pst-node}' which allows you to
% use the powerful PSTricks. If you don't have this package on your
% computer you will have to install it.

% A4 will have to be redone.
% Thanks to Bill Purvis (bil@beeb.net, http://bil.members.beeb.net)
% for help with setting up A4 paper the first time ...

% ******************************************************************************
% PAGE LAYOUT

% Cards are 3.5x2 inches.
% The Letter size paper is 8.5x11.
% Together these can be used to precisely lay out 10 cards as follows.

% Horizontal Card Layout:
% | edge   | card 1 | card 2 | edge   |
% | 0.75in | 3.5in  | 3.5in  | 0.75in | = 8.5 wide

% Vertical Card Layout:
% ------------------ top of paper
%   0.5 inch space
% ------------------ top of card 5
%   2.0 inch card 5
% ------------------ top of card 4
%   2.0 inch card 4
% ------------------ top of card 3
%   2.0 inch card 3
% ------------------ top of card 2
%   2.0 inch card 2
% ------------------ top of card 1
%   2.0 inch card 1
% ------------------ bottom of card 1
%   0.5 inch space
% ------------------ bottom of paper

% http://en.wikibooks.org/wiki/LaTeX/Page_Layout
% \usepackage[top=tlength, bottom=blength, left=llength, right=rlength]{geometry}
% the 0000 fails to be exactly aligned on the page!
% View the page in Adobe Acrobat with command R to get a ruler.
% The ruler doesn't match exactly ...

\usepackage[margin=0.0in]{geometry}

% ******************************************************************************

% Set up pstricks
% http://www.ctan.org/tex-archive/graphics/pstricks/doc/pstnews1-14.pdf
% or section 3.4 in
% http://www.ctan.org/tex-archive/graphics/pstricks/README
\usepackage{pstricks} % allows using PSTricks
\usepackage{pst-node} % nodes in pst
\usepackage{graphics}
\usepackage{tgschola}
\usepackage[T1]{fontenc}
\graphicspath{ {/home/rtc/texlive/} }
% \usepackage{color} % see above - it's loaded by pstricks.

% ******************************************************************************

% Adjust exact lower left corner for your printer
\newcommand{\xadjust}{0.0cm}
% \newcommand{\yadjust}{0.3cm}
\newcommand{\yadjust}{0.25cm}

\pagestyle{empty} % removes page numbers

% ******************************************************************************

%\multirput*[refpoint]{angle}(x0,y0)(x1,y1){int}{stuff }

\begin{document}
\noindent
\setlength{\unitlength}{1in}
\rput[bl]{0}(0.75in,-10.50in){% global set of lower left corner
 \rput[bl]{0}(\xadjust,\yadjust){%
  % show the psgrid to see where the cards will be placed
  %\psgrid[subgriddiv=1,xunit=3.5in,yunit=2.0in]%
  %       (0,0)(-20,-20)(20,20) % provide a grid from PSTricks
  \multirput[bl](0,0)(3.5in,0.0in){2}{%
  \multirput[bl](0,0)(0.0in,2.0in){5}{%
   \makebox(3.5,2.0){% x,y size of box, inches
    \shortstack[l]{%
     \includegraphics[scale=0.025]{logo}\\
\textsc{\huge{Rachael Carlson}}\\
\emph{\tiny{music}}
\vspace*{5mm}
% \bigskip\break
\hrule
\vfill
\tiny{Milwaukee, WI \hfill (612) 616-4422} \\ 
\tiny{rachaelcarlson.us} \hfill \tiny{rtc@sleeplimited.org} % Your address and contact information
% use \\ to separate lines in guts
    }%
   }%
  }%
  }%
 }%
}
\end{document}

% \psgrid(0,0)(-20,-20)(20,20) % provide a grid from PSTricks
% \psgrid[subgriddiv=1,unit=1in](0,0)(-20,-20)(20,20) %provide a grid from PSTricks
% The basic syntax for the Latex picture environment is.
% \begin{picture}(width,height)(x-offset,y-offset)
% x-offset of 1 moves LEFT 1 inch!
% y-offset of 1 moves DOWN 1 inch!
%
% \begin{picture}(8.5,11)(0.0,0.3937) % revised for 5 vertical per page
% 5 cards vertical per page
% \begin{picture}(8.5,11)(-0.20,+0.50)
% \begin{picture}(8.5,11)(-0.20,+0.58) % ok for home too low for work
\begin{picture}(8.5,11)(-0.20,+0.50) %  ok for home
  \thicklines
  % 2 is the number of cards horizontally

%\vspace{10.0in}\hspace{0in}% start at lower left point
%  \multiput(0,0)(3.7,0.0){2}{ % horizontal (x) motion
% revise for minimalist gap between cards 3.5" + gap:
% \multiput(0,0)(3.54,0.0){2}{ % horizontal (x) motion
% 2009 Nov 26: no gap:
  \multiput(0,0)(3.50,0.0){2}{ % horizontal (x) motion
% numbers for 4 per page:
%     \multiput(0,0)(0.0,2.15625){\verticalcards}{ % vertical (y) motion
% numbers for 5 per page:
%     \multiput(0,0)(0.0,2.1){\verticalcards}{ % vertical (y) motion
%
% revise for minimalist gap between cards, 2.0"+gap:
%    \multiput(0,0)(0.0,2.04){\verticalcards}{ % vertical (y) motion
% 2009 Nov 26: no gap:
    \multiput(0,0)(0.0,2.00){\verticalcards}{ % vertical (y) motion
%       \framebox(3.5,2){ % x,y size of box, inches
        \makebox(3.5,2){ % x,y size of box, inches
           \shortstack[l]{
\includegraphics[scale=0.025]{logo}\\
\textsc{\huge{Rachael Carlson}}\\
\emph{\tiny{music}}
\vspace*{5mm}
% \bigskip\break
\hrule
\vfill
\tiny{Milwaukee, WI \hfill (612) 616-4422} \\ 
\tiny{rachaelcarlson.us} \hfill \tiny{rtc@sleeplimited.org} % Your address and contact information

           } % end shortstack
        } % end makebox
%       } % end framebox
     } % end multiput
  } % end multiput
\end{picture}
\end{document}
